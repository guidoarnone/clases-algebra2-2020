\documentclass{beamer}

\usepackage[utf8]{inputenc}
\usepackage{color}
\usepackage{amsmath}
\usepackage{amsthm}
\usepackage{amsfonts, amssymb}
\usepackage{mathrsfs}
\usepackage{tikz-cd}
\usepackage{graphics}
\usepackage{epstopdf}
\usepackage{lmodern}
\usepackage{bbm} % black board numbers
\usepackage{listings}
\usepackage{hyperref}
%\usepackage{times}
%\usepackage[T1]{fontenc}


\mode<presentation>{

	\usetheme{Boadilla} 
	\usecolortheme{whale}
	%\usefonttheme{professionalfonts}
}





\newcommand{\tq}{\mathrel{{\ensuremath{\: : \: }}}}

\newcommand{\st}{\ensuremath{\mathrm{st}}}
\newcommand{\starabierto}{\ensuremath{\stackrel{\circ}{\mathrm{st}}}}
\newcommand{\lk}{\ensuremath{\mathrm{lk}}}

\newcommand{\Hom}{\ensuremath{\mathrm{Hom}}}
\newcommand{\Ext}{\ensuremath{\mathrm{Ext}}_{\mathbb{Z}}}
\newcommand{\Tor}{\ensuremath{\mathrm{Tor}}^{\mathbb{Z}}}

\newcommand{\imagen}{\ensuremath{\mathrm{Im}}}
\newcommand{\tr}{\ensuremath{\mathrm{tr}}}
\newcommand{\rk}{\mathrm{rk}}

\newcommand{\id}{\mathbbm{1}}
\newcommand{\Sq}{\ensuremath{\mathrm{Sq}}}
\newcommand{\deficiency}{\mathrm{def}}
\newcommand{\Fix}{\mathrm{Fix}}
\def\action{\curvearrowright}
\newcommand{\LS}[2]{\left(\dfrac{#1}{#2}\right)}


\newcommand{\Out}{\mathrm{Out}}
\newcommand{\Outdiag}{\mathrm{Outdiag}}
\newcommand{\Inn}{\mathrm{Inn}}
\newcommand{\Aut}{\mathrm{Aut}}


\def\wt{\widetilde}
\def\ms{\medskip}
\def\bs{\bigskip}


\def\N{\mathbb{N}}
\def\Z{\mathbb{Z}}
\def\Q{\mathbb{Q}}
\def\R{\mathbb{R}}
\def\C{\mathbb{C}}
\def\H{\mathbb{H}}

\def\GL{\mathrm{GL}}
\def\SO{\mathrm{SO}}
\def\sg{\mathrm{sg}}
\def\det{\mathrm{det}}


\def\sharp{\#}

\newcommand{\catname}[1]{{\normalfont\textbf{#1}}}
\newcommand{\Set}{\catname{Set}}
\newcommand{\Top}{\catname{Top}}
\newcommand{\Ab}{\catname{Ab}}
\newcommand{\Grp}{\catname{Grp}}
\newcommand{\op}{\catname{op}}

\def\1{\mathbbm{1}}
%\beamerdefaultoverlayspecification{<+->} % itemize automatico
\newtheorem{teorema}{Teorema} [section]
\newtheorem{teorema1}[teorema]{Teorema 1}
\newtheorem{teorema2}[teorema]{Teorema 2}
\newtheorem{teorema3}[teorema]{Teorema 3}
\newtheorem{pregunta}[teorema]{Pregunta}
\newtheorem{definicion}[teorema]{Definici\'on} 
\newtheorem{lema}[teorema]{Lema}
\newtheorem{corolario}[teorema]{Corolario} 
\newtheorem{ejemplo}[teorema]{Ejemplo}
\newtheorem{ejercicio}[teorema]{Ejercicio}
\newtheorem{ejemplos}[teorema]{Ejemplos}
\newtheorem{observacion}[teorema]{Observaci\'on}
\newtheorem{proposicion}[teorema]{Proposici\'on}
\newtheorem{conjetura}[teorema]{Conjetura}



\setbeamertemplate{navigation symbols}{}
\setbeamertemplate{footline}
        {
      \leavevmode
      \hbox{
      \begin{beamercolorbox}[wd=.333333\paperwidth,ht=2.25ex,dp=1ex,center]{author in head/foot}%
        \usebeamerfont{author in head/foot}\insertshortauthor
      \end{beamercolorbox}
      \begin{beamercolorbox}[wd=.333333\paperwidth,ht=2.25ex,dp=1ex,center]{title in head/foot}%
        \usebeamerfont{title in head/foot}\insertshorttitle
      \end{beamercolorbox}
      \begin{beamercolorbox}[wd=.333333\paperwidth,ht=2.25ex,dp=1ex,center]{date in head/foot}%
        \usebeamerfont{date in head/foot}\insertshortdate

    %#turning the next line into a comment, erases the frame numbers
        %\insertframenumber{} / \inserttotalframenumber\hspace*{2ex} 

      \end{beamercolorbox}
			}
      \vskip0pt
    }

\newenvironment{psmallmatrix}
  {\left(\begin{smallmatrix}}
  {\end{smallmatrix}\right)}

\setbeamercolor{alerted text}{fg=blue!100!black!200!}


% Author, Title, etc.
\title[Álgebra II Práctica (clase 5)] {Álgebra II Práctica (clase 5)}
\author[G. Arnone]{Guido Arnone}
\institute[]{Universidad de Buenos Aires}
\date[2020/04/28]{28 de Abril de 2020}


% The main document
\begin{document}

{
\setbeamertemplate{footline}{}
\begin{frame}
\titlepage
\end{frame}
}

\section{Prerrequisitos}
\begin{frame}{Prerrequisitos}
Para leer estas diapositivas se recomienda haber leído el apunte teórico hasta la Sección 1.6.

\medskip

%\tableofcontents
\end{frame}

\begin{frame}{Sistemas de Generadores}
Vamos a relacionar las definiciones de grupo normal y cociente con la noción de sistema de generadores de un grupo.

\medskip
\onslide<2-> 
Recordemos que dado un grupo $G$, un subconjunto $S \subset G$ es un \alert{sistema de generadores} de $G$ si todo elemento de $G$ se puede escribir como un producto de elementos de $S$ o sus inversos. \onslide<3-> Concretamente, si $g \in G$ entonces existen $s_1, \dots, s_n \in S$ y $\varepsilon_1, \dots, \varepsilon_n \in \{-1,1\}$ tales que
\[
g = s_1^{\varepsilon_1} \cdots s_n^{\varepsilon_n}.
\]

\onslide<4-> Equivalentemente, es $G = \langle S \rangle = \bigcap_{S \subset H \leq G}H$.
\onslide<5-> Decimos también que $G$ \alert{está generado} por $S$. \onslide<6-> Por ejemplo, el grupo $S_n$ está generado por las transposiciones y el diedral $D_n$ por los elementos que denotamos $r$ y $s$.
\end{frame}


\begin{frame}{Sistemas de Generadores - Grupos Normales}
Trabajar con generadores nos permite dar información sobre todo el grupo $G$ analizando sólo lo que sucede para algunos elementos. Por ejemplo,

\onslide<2->\begin{proposicion} Sea $G$ un grupo y $S$ un conjunto de generadores. Un subgrupo $H$ de $G$ es normal si y solo si $sHs^{-1} \subset H$ y $s^{-1}Hs \subset H$ para todo $s \in S$.
\end{proposicion}
\onslide<3->\begin{proof}[Demostración] Si $H$ es normal, ya sabemos que $gHg^{-1} \subset H$ para \alert{todo} $g \in G$, lo que debemos ver es la recíproca. \onslide<4-> Tomemos $h \in H,g \in G$ y escribamos $g = s_1^{\varepsilon_1} \cdots s_n^{\varepsilon_n}$. \onslide<5-> En estos términos, debe ser $g^{-1} = s_n^{-\varepsilon_n} \cdots s_1^{-\varepsilon_1}$. \onslide<6-> Ahora, por hipótesis $h_1 = s_n^{\varepsilon_n}hs_n^{-\varepsilon_n}$ es un elemento de $H$. \onslide<7-> Del mismo modo $h_2 = s_{n-1}^{\varepsilon_{n-1}}h_1s_{n-1}^{-\varepsilon_{n-1}}$ también está en $H$.\onslide<8-> Repetimos el proceso hasta llegar a que
\[
ghg^{-1} = s_1^{\varepsilon_1} \cdots s_n^{\varepsilon_n}h s_n^{-\varepsilon_n} \cdots s_1^{-\varepsilon_1}
\]
pertenece a $H$, como queríamos ver.
\end{proof}
\end{frame}
  
\begin{frame}{Sistemas de Generadores - Grupos Normales (cont.)}
\begin{ejercicio} El resultado anterior es falso si pedimos solamente que $sHs^{-1} \subset H$ para todo $s \in S$. Justificar por qué la demostración anterior no es correcta si sólo tenemos esta hipótesis.
\end{ejercicio}
\onslide<2->Algunas consecuencias inmediatas:
\begin{itemize}
\onslide<3->\item Un subgrupo $H$ es normal en $S_n$ si y sólo si $\tau H \tau^{-1} \subset H$ para toda \alert{transposición} $\tau \in S_n$.
\onslide<4-> \item Para decidir si un subgrupo $H$ de $D_n$ es normal, alcanza ver que $sHs \subset H$ y $rHr^{-1}, r^{-1}Hr \subset H$.
\end{itemize}
\end{frame}
  
\begin{frame}{Sistemas de Generadores - Cocientes}
Si $H$ es un subgrupo normal de un grupo $G$, sabemos que $G/H$ es un grupo, y por lo tanto tiene sentido hablar de un sistema de generadores para $G/H$. \onslide<2-> A partir de generadores de $G$ podemos conseguir generadores de $G/H$,

\onslide<3->\begin{proposicion} Sea $G$ un grupo y $H \triangleleft G$. Notamos $\pi : g \mapsto [g] = gH \in G/H$ a la proyección canónica. Si $S$ es un conjunto de generadores de $G$, entonces $T := \pi(S) = \{[s] : s \in S\}$ es un conjunto de generadores de $G/H$.
\end{proposicion}
\onslide<4->\begin{proof}[Demostración] Tomemos $x \in G/H$. Sabemos que es de la forma $x= [g] = gH$ para algún $g \in G$. \onslide<5->Como $S$ genera a $G$, es $g = s_1^{\varepsilon_1} \cdots s_n^{\varepsilon_n}$ para ciertos $\{s_i\}_{i=1}^n \subset S, \{\varepsilon_i\}_{i=1}^n \subset \{-1,1\}$. \onslide<6-> Aplicando $\pi$ a ambos lados de la igualdad, resulta $[g] = [s_1^{\varepsilon_1} \cdots s_n^{\varepsilon_n}] = [s_1]^{\varepsilon_1} \cdots [s_n]^{\varepsilon_n}$.
\end{proof}
\end{frame}

\begin{frame}{Sistemas de Generadores - Cocientes (cont.)}
También podemos conseguir generadores de $G$ a partir de generadores de $H$ y $G/H$. Concretamente,
\onslide<2-> \begin{proposicion} Sea $G$ un grupo y $H$ un subgrupo normal. Si tenemos $S,T \subset G$ tales que $S$ genera a $H$ y $\pi(T)$ genera a $G/H$, entonces $S \cup T$ genera a $G$.
\end{proposicion}
\onslide<3-> \begin{proof}[Idea de la demostración]
\begin{itemize}
\onslide<4-> \item Tomemos $g \in G$. Por hipótesis $[g]$ se escribe como $[g] = [t_1]^{\varepsilon_1} \cdots [t_n]^{\varepsilon_n}$ $= [t_1^{\varepsilon_1} \cdots t_n^{\varepsilon_n}]$ para ciertos $t_i \in T$.
\onslide<5-> \item Esta igualdad nos dice que existe $h \in H$ tal que $g = t_1^{\varepsilon_1} \cdots t_n^{\varepsilon_n}h$.
\onslide<6-> \item Por otro lado debe ser $h = s_1^{\delta_1} \cdots s_n^{\delta_m}$ con $s_j \in S$ así que, en definitiva, es $g = t_1^{\varepsilon_1} \cdots t_n^{\varepsilon_n}s_1^{\delta_1} \cdots s_n^{\delta_m}$.
\end{itemize}  
\end{proof}
\end{frame}

\begin{frame}{Sistemas de Generadores - Una Aplicación}
Antes de seguir, veamos una aplicación:

\medskip

\onslide<2-> Dijimos anteriormente que si $n \geq 3$, entonces $S_n/A_n = \{[1],[\tau]\}$ con $\tau$ cualquier transposición. Este cociente es un grupo cíclico generado por $[\tau]$. 

\medskip

\onslide<3->Por otro lado, el grupo alternante está generado por los $3$-ciclos (\href{https://bit.ly/3cONlKe}{\alert{esta}} respuesta en Zulip incluye una demostración). \onslide<4->El resultado que probamos nos dice que $S_n$ está generado por los $3$-ciclos y una única transposición (y no importa cuál elijamos!).

\medskip
\end{frame}

\begin{frame}{El primer teorema de isomorfismo}
\alert{Recordar:} si $f : G \to G'$ es un morfismo de grupos y $K \triangleleft G$ un subgrupo normal de $G$ contenido en $\ker f$, existe un único morfismo de grupos 
\[
\overline{f} : [g] \in G/K \mapsto f(g) \in G'
\]
tal que $\overline{f}\pi = f$, donde $\pi: G \to G/K$ es la proyección canónica. Además, se tiene $\mathsf{im} f = \mathsf{im} \overline{f}$. 
\medskip

\onslide<2-> Cuando $K = \ker f$, el morfismo $\overline{f}$ es inyectivo y entonces es un isomorfismo con su imagen, por lo que
\[
G/\ker f \simeq \mathsf{im} f.
\]
\end{frame}  

\begin{frame}{El primer teorema de isomorfismo}
En estos términos, los cocientes que vimos hace unas clases se pueden caracterizar \alert{como grupos}:
\begin{itemize}
\onslide<2-> \item $\sg : S_n \to G_2 \rightsquigarrow S_n/A_n \simeq G_2$
\onslide<3-> \item $r_n : \Z \to \Z_n \rightsquigarrow \Z/n\Z \simeq \Z_n$, con $r_n$ la función que devuelve el resto en la división por $n \in \N$.
\onslide<4-> \item $p : (x,y) \in \R^2 \mapsto (0,y) \in \R \rightsquigarrow \R^2/\R \oplus 0 \simeq \R$. 
\end{itemize}
\medskip \medskip
\onslide<5-> Veamos otro ejemplo, para el cual necesitaremos primero algunas definiciones.
\end{frame}  

\begin{frame}{El grupo afín}

Una función $T : \R^n \to \R^n$ se dice \alert{afín} si es de la forma $T(x) = Ax + b$ con $b \in \R^n$ y $A \in \mathsf{M}_n\R$. Notaremos $T = A + b$. \onslide<2->Intuitivamente, estamos considerando funciones lineales que "olvidan el origen", ya que permitimos trasladar.

\onslide<3->\begin{ejercicio} Probar que:
\begin{itemize}
\onslide<4-> \item Una transformación afín $T = A + b$ es inversible si y sólo si $A$ lo es. En tal caso, su inversa es también una transformación afín.
\onslide<5-> \item La composición de transformaciones afines resulta una transformación afín. 
\end{itemize}
\end{ejercicio}

\onslide<6->El ejercicio anterior justifica que 
\begin{align*}
Aff_n(\R) :&= \{T : \R^n \to \R^n : \text{$T$ es afín e inversible} \}\\
&= \{T : \R^n \to \R^n \mid T(x) = Ax + b, b \in \R^n, A \in \mathsf{GL}_n(\R)\}
\end{align*}
es un grupo con la composición. Lo llamamos el \alert{grupo afín} de $\R^n$.
\end{frame}

\begin{frame}{El grupo afín (cont.)}
En $Aff_n(\R)$ hay dos subgrupos distinguidos:
\begin{itemize}
\onslide<2-> \item Las funciones lineales inversibles $H := \{T \in Aff_n(\R) : \text{$T$ es lineal}\}$.
\onslide<3-> \item Las traslaciones, 
\[
K := \{T \in Aff_n(\R) \mid T(x) = x+b, b \in \R^n\} = \{I + b : b \in \R^n\}.
\]
\end{itemize}

\onslide<4-> \begin{ejercicio} Probar que:
\begin{itemize}
\onslide<5-> \item Existen isomorfismos $H \simeq \mathsf{GL}_n(\R)$ y $K \simeq \R^n$.
\onslide<6-> \item El subgrupo $K$ es normal en $Aff_n(\R)$, pero $H$ no. 
\end{itemize}
\end{ejercicio}
\end{frame}

\begin{frame}{El grupo afín (cont.)}
Observemos también que todo elemento $S(x) = Ax + b$ del grupo afín se escribe como $S = TL$ con $T(x) = x+b$ una traslación y $L(x) = Ax$ una función lineal inversible. De hecho,

\onslide<2->\begin{ejercicio}[para más adelante!] Probar que $Aff_n(\R) \simeq \R^n \rtimes \mathsf{GL}_n(\R)$.
\end{ejercicio}
\medskip

\onslide<3->Volvamos al primer teorema de isomorfismo: sabemos que las traslaciones son un grupo normal. \onslide<4->Intuitivamente, identificando transformaciones afines si difieren en una traslación deberíamos recuperar "su parte lineal". \onslide<5->Es decir, deberíamos tener que $Aff_n(\R)/K \simeq \mathsf{GL}_n(\R)$. Veámoslo.
\end{frame}

\begin{frame}{El grupo afín (cont.)}
En base a lo anterior, consideramos la función
\begin{align*}
\Lambda \colon &Aff_n(\R) \rightarrow \mathsf{GL}_n(\R) \\
&A+b \longmapsto A 
\end{align*}
\onslide<2-> Si $T = A + b$ y $S = C + d$, entonces
\[
T \circ S(x) = T(Cx+d) = ACx + Ad + b
\]
y $\Lambda(TS) = AC = \Lambda(T)\Lambda(S)$. \onslide<3->Por lo tanto $\Lambda$ es un morfismo de grupos. Además $\Lambda$ es sobreyectivo, ya que $\Lambda(A) = A$ para toda $A$ inversible, y
\[
\ker \Lambda = \{I+b : b \in \R^n\} = K.
\] 

\onslide<4->Por el primer teorema de isomorfismo, obtenemos efectivamente que
\[
Aff_n(\R)/K \simeq \mathsf{GL}_n(\R).
\]
\end{frame}

\begin{frame}{Subgrupos de un Cociente}
Para terminar, caractericemos a los subgrupos de un cociente. Sea $G$ un grupo y $H$ un subgrupo normal en $G$. \onslide<2->Si aplicamos la Proposición 1.4.24 del apunte teórico a la proyección $\pi : G \to G/H$, obtenemos una correspondencia biyectiva
\[
\{H \subset K : \text{ $K$ subgrupo de $G$}\} \leftrightarrow \{L \subset G/H : \text{$L$ subgrupo de $G/H$}\}
\]
que envía $K$ a $\pi(K)$ y $L$ a $\pi^{-1}(L)$. Además, la correspondencia envía subgrupos grupos normales a subgrupos normales.
\medskip

\onslide<3-> Relacionando esto con el ejemplo anterior, los subgrupos de $Aff_n(\R)$ que contienen a las traslaciones se identifican con los subgrupos de $Aff_n(\R)/K$.
\end{frame}

\begin{frame}{Subgrupos de un Cociente (cont.)}
Por otro lado, pudimos caracterizar el cociente como un grupo conocido, y bajo el isomorfismo $\overline{\Lambda} : Aff_n(\R)/K \to \mathsf{GL}_n(\R)$ los subgrupos de $Aff_n(\R)/K$ son de la forma $\overline{\Lambda}^{-1}(L)$ con $L$ un subgrupo de $\mathsf{GL}_n(\R)$. 
\medskip

\onslide<2-> Juntando ambas correspondencias, vemos que todo subgrupo de $Aff_n(\R)$ que contiene a las traslaciones es de la forma
\[
H_L := \{Ax+b : b\in \R^n, A \in L\}
\]
con $L \leq \mathsf{GL}_n(\R)$.
\end{frame}

\begin{frame}{Ejercicios}
Pueden relacionar todos los conceptos que vimos con los siguientes ejercicios,

\onslide<2->\begin{ejercicio} ¿Cuántos subgrupos de $D_{7981326}$ que contienen a $H := \langle r^2 \rangle$ hay?
\end{ejercicio}

\onslide<3->\begin{ejercicio} Un grupo $G$ se dice \alert{simple} si $G \neq \{1\}$ y sus únicos subgrupos normales son $\{1\}$ y $G$. 
\begin{itemize}
\item[(i)] Sea $H$ un subgrupo normal en un grupo $G$. Probar que $G/H$ es simple si y sólo si el único subgrupo normal que contiene propiamente a $H$ es $G$.
\item[(ii)] Probar que un grupo abeliano finito $G$ es simple si y sólo si $G \simeq \Z_p$ con $p$ primo. Concluir que en todo grupo abeliano finito  $G \neq \{1\}$ existe un subgrupo $H$ tal que $G/H \simeq \Z_p$, para algún $p$ primo.
\end{itemize}
\end{ejercicio}
\end{frame}
 
\end{document}